\documentclass[11pt]{scrartcl}
\usepackage[utf8]{inputenc} % Kodierung der Textdatei mit Sonderzeichen
\usepackage[ngerman]{babel} % Sprache fuer Inhaltsverzeichnis etc.
\usepackage{amssymb} % Mathematische Symbole
\usepackage{amsmath} % Mehr mathematische Konstrukte
\usepackage{graphicx} % Um Bilder einbinden zu koennen
\usepackage{float} % fuer \begin{figure}[H]
\usepackage{icomma} % laesst das Komma als Dezimaltrennzeichen interpretieren
\usepackage{fix-cm} % für die große Titelschrift
\usepackage[pdftex]{hyperref} % Hyperlinks im Dokument
\hypersetup{colorlinks=true, linkcolor=black, citecolor=black, filecolor=black, urlcolor=black, pdftitle={LED-Spektrometer - Projektpraktikum 09/10 Gruppe 5}}


\newcommand{\unit}[1]{\ensuremath{\,\mathrm{#1}}} % Einheiten schreiben sich immer aufrecht!
\newcommand{\degr}{\ensuremath{^\circ}}
\newcommand{\cel}{\ensuremath{\degr\mathrm{C}}}
\newcommand{\dif}{\ensuremath{\mathrm{d}}}
\newcommand{\pdif}[2]{\ensuremath{\frac{\partial#1}{\partial#2}}}
\newcommand{\ee}[1]{\ensuremath{\cdot 10^{#1}}}
\newcommand{\hypref}[2]{\hyperref[#2]{{#1}~\ref{#2}}}

\setlength{\parindent}{1em}
\setlength{\parskip}{0.5\baselineskip}


\title{LED-Spektrometer - Gruppe 5 WS 09/10, Projektpraktikum der Uni Erlangen}
\date{07.12.2009 -- 15.01.2010}
\author{Michele Collodo, Andreas Glossner, Karl-Christoph G\"odel, Bastian Hacker, Maria Obst, Alexander Wagner, David Winnekens}



\begin{document}
\sloppy % laesst Latex nicht ueber den Rand rausschreiben
\thispagestyle{empty}
\large{Projektpraktikum WS 09/10}
\hfill
\raisebox{-1.4cm}{\includegraphics[width=5cm]{images/fau.pdf}}
\\[8\baselineskip]
\begin{center}
{\fontsize{36}{54}\textbf{LED-Spektrometer}}
\\[2\baselineskip]
{\Large 07.12.2009 -- 15.01.2010}
\\[7\baselineskip]
{\huge\textbf{PPG 5}}
\\[0.5\baselineskip]
{\large\textbf{
Michele Collodo,
Andreas Glossner,\\
Karl-Christoph G\"odel,
Bastian Hacker,\\
Maria Obst,
Alexander Wagner,
David Winnekens}\\
Tutor: Xiaoyue Jin}
\vfill



\small{\url{http://pp.physik.uni-erlangen.de/groups/ws0910/ppg5/ppg5\_start.html}}
\end{center}
\newpage



\tableofcontents
\vfill



\begin{abstract}
Bla
\end{abstract}
\newpage


\section{Grundgedanke des Versuchs}
%% Allg. was wir machen wollen, etwas Theorie 	Karl
%% Vergleich mit den Emissionsspekren}

\section{Messung der Absorptionsspektren}

\subsection{Versuchsaufbau}
%%\subsubsection{Lichtquelle, LED vs. Halogen} %%subsubsections wohl lieber weglassen, aber zur Meinungsbildung erst noch drinnen gelassen.					Axi
\textit{(Lichtquelle, LED vs. Halogen ...)}\\  %%David
%%\subsubsection{Gitter, Prinzip, optischer Strahlengang} %%optischer Strahlengang ist unten eingebaut. Deshalb auch die Idee, die subsubsections weg zu lassen.
Zur Erzeugung eines Spektrums gibt es diverse Möglichkeiten. Es stellte sich aber heraus, dass die Verwendung eines Prismas aufgrund der geringen Winkelausdehnung des erzeugten Spektrums für uns wenig lohnenswert ist. Ebenso mussten wir trotz langen Experimentierens und unter Verwendung verschiedener Linsenanordnungen auch auf ein Transmissionsgitter verzichten, da die Intensität des Spektrums nicht stark genug war. Folglich griffen wir auf ein holographisches Reflexionsgitter zurück.\\%%Gitterkonstante und Winkelausdehnung des Spektrums angeben
%%\subsubsection{Winkelmessung} %%--- Eher "Wellenlängenbestimmung" oder dergleichen?! Da wir dadurch ja die momentan gemessene Wellenlänge bestimmen möchten
Für die Auswertung der Messdaten ist es von grundlegender Bedeutung, zu wissen welcher Wellenlänge die gemessene Intensität zuzuordnen ist. Um dies zu erreichen wurde folgendes Vorgehen gewählt: Durch die Verwendung eines Refelxionsgitters und dessen Justierung derart, dass der Strahlengang in einer zum optischen Tisch parallelen Ebene verläuft, lässt sich die Wellenlänge durch Formel XXX leicht in den Winkel zwischen ein- und ausfallendem Strahl umrechnen. Da ein manuelles Auslesen der Intensität und dessen zugehörigem Winkel ein aussichtsloses Unterfangen darstellen würde, benutzten wir ein Drehpotentiometer zur automatischen Auswertung des momentanen Winkels. Dieses wurde so auf dem Tisch positioniert, dass die Drehachse möglichst genau unter dem Mittelpunkt des Gitters ist, also dem Punkt, an welchem sich ein- und ausfallender Strahl treffen. Desweiteren wurde der Dreharm, an wessen äußerstem Ende sich der Halter für die LEDs befindet, mit der Drehachse verbunden, sodass das Schwenken des Armes den Widerstand des Potentiometeres ändert. Anschließen wurde in $2^\circ$-Intervallen der Winkel und die Spannung am Potentiometer notiert. Wiederholtes Messen dieser Spannungs-Winkel-Abhängigkeit zeigte eine sehr gut Reproduzierbarkeit. Lediglich beim Wechsel des Drehsinns zeigte sich ein Offset, sodass entschlossen wurde alle Messungen immer mit Drehung im Uhrzeigersinn durchzuführen.\\
Somit konnten wir die momentane Spannung am Potentiometer auslesen lassen, was uns den Winkel gibt und somit die Wellenlänge berechnen lässt.


\subsection{Reduktion der Spektren}
%% Fit der Winkelfunktion
%% Entfaltung                 Basti





\section{Bau des Spektrometers}
\subsection{Funktionsprinzip Übersicht}

\subsection{Elektronik}
%% Funktionsprinzip OpAmps etc.			Michele, Maria
%% Schaltplan

\subsection{Auswertungsprogramm}
% Software: Berechnung der Spektren mit unserem Gerät
%% Funktionsübersicht und Erklärung der Bedienung
%% Algorithmik zur Auswertung					BAsti



\newpage
\section{Autorenverzeichnis}
\begin{tabular}{|l|l|}
\hline
\emph{Autor} & \emph{Kapitel}\\
\hline
Michele Collodo & \\
Andreas Glossner & \\
Karl-Christoph G\"odel & \\
Bastian Hacker & \\
Maria Obst & \\
Alexander Wagner & \\
David Winnekens &  \\
Wickie Pedia & Recherchen \\
\hline
\end{tabular}
\end{document}

