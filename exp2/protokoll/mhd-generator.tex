\documentclass[11pt]{scrartcl}
\usepackage[utf8]{inputenc} % Kodierung der Textdatei mit Sonderzeichen
\usepackage[ngerman]{babel} % Sprache fuer Inhaltsverzeichnis etc.
\usepackage{amssymb} % Mathematische Symbole
\usepackage{amsmath} % Mehr mathematische Konstrukte
\usepackage{graphicx} % Um Bilder einbinden zu koennen
\usepackage{float} % fuer \begin{figure}[H]
\usepackage{icomma} % laesst das Komma als Dezimaltrennzeichen interpretieren
\usepackage[pdftex]{hyperref} % Hyperlinks im Dokument
\hypersetup{colorlinks=true, linkcolor=black, citecolor=black, filecolor=black, urlcolor=black, pdftitle={MHD-Generator - Projektpraktikum 09/10 Gruppe 5}}


\newcommand{\unit}[1]{\ensuremath{\,\mathrm{#1}}} % Einheiten schreiben sich immer aufrecht!
\newcommand{\degr}{\ensuremath{^\circ}}
\newcommand{\cel}{\ensuremath{\degr\mathrm{C}}}
\newcommand{\dif}{\ensuremath{\mathrm{d}}}
\newcommand{\pdif}[2]{\ensuremath{\frac{\partial#1}{\partial#2}}}
\newcommand{\ee}[1]{\ensuremath{\cdot 10^{#1}}}


\setlength{\parindent}{1em}
\setlength{\parskip}{0.5\baselineskip}


\title{MHD-Generator - Gruppe 5 WS 09/10, Projektpraktikum der Uni Erlangen}
\date{02.11.2009 -- 28.11.2009}
\author{Michele Collodo, Andreas Glossner, Karl-Christoph G\"odel, Bastian Hacker, Maria Obst, Alexander Wagner, David Winnekens}



\begin{document}
\sloppy % laesst Latex nicht ueber den Rand rausschreiben
\thispagestyle{empty}
\large{Projektpraktikum WS 09/10}
\hfill
\raisebox{-1.4cm}{\includegraphics[width=5cm]{images/fau.pdf}}
\\[8\baselineskip]
\begin{center}
{\Huge\textbf{MHD-Generator}}
\\[2\baselineskip]
{\Large 02.11.2009 -- 28.11.2009}
\\[6\baselineskip]
{\huge\textbf{PPG 5}}\\[0.5\baselineskip]
{\large\textbf{
Michele Collodo,
Andreas Glossner,\\
Karl-Christoph G\"odel,
Bastian Hacker,\\
Maria Obst,
Alexander Wagner,
David Winnekens}\\
Tutor: Xiaoyue Jin}
\vfill



\small{\url{http://pp.physik.uni-erlangen.de/groups/ws0910/ppg5/ppg5\_start.html}}
\end{center}
\newpage



\tableofcontents
\vfill



\begin{abstract}
dumdidum
\end{abstract}
\newpage

\section{Einleitung}
Faraday hats wohl 1832 entdeckt.

\section{Theorie}

\section{Aufbau}

\section{Messungen und Ergebnisse}

\subsection{Leitfähigkeit von Salzwasser}
Die interessantesten Messungen gibts hier

\subsection{Magnetfeld}
Wie hängt das Magnetfeld vom Strom, Geometrie etc. ab?

\subsection{Generatorspannung und -strom}

\subsection{Wassergeschwindigkeit und Druck}
... Wirkungsgrad gibts keinen, weil die Druckmessung zu schwierig war.

\subsection{Umkehrung des Effekts, Nutzung als Wasserpumpe}
Wird sogar als Antriebstechnik für Schiffe getestet (Magnetohydrodynamischer Antrieb bei Wikipedia)

\section{Fazit}

\end{document}

