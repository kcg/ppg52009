\documentclass[11pt]{scrartcl}
\usepackage[utf8]{inputenc} % Kodierung der Textdatei mit Sonderzeichen
\usepackage[ngerman]{babel} % Sprache fuer Inhaltsverzeichnis etc.
\usepackage{amssymb} % Mathematische Symbole
\usepackage{amsmath} % Mehr mathematische Konstrukte
\usepackage{graphicx} % Um Bilder einbinden zu koennen
\usepackage{float} % fuer \begin{figure}[H]
\usepackage{icomma} % laesst das Komma als Dezimaltrennzeichen interpretieren
\usepackage[pdftex]{hyperref} % Hyperlinks im Dokument
\hypersetup{colorlinks=true, linkcolor=black, citecolor=black, filecolor=black, urlcolor=black}


\newcommand{\unit}[1]{\ensuremath{\,\mathrm{#1}}} % Einheiten schreiben sich immer aufrecht!
\newcommand{\degr}{\ensuremath{^\circ}}
\newcommand{\cel}{\ensuremath{\degr\mathrm{C}}}
\newcommand{\dif}{\ensuremath{\mathrm{d}}}
\newcommand{\pdif}[2]{\ensuremath{\frac{\partial#1}{\partial#2}}}
\newcommand{\ee}[1]{\ensuremath{\cdot 10^{#1}}}


\setlength{\parindent}{0pt}
\setlength{\parskip}{0.5\baselineskip}


\title{Besprechungsprotokoll - PPG 5/2009}
\date{09.11.2009, 14.30 Uhr}
\author{Protokollschreiber: Alexander Wagner}



\begin{document}
\maketitle
\pagestyle{empty}
\section{Anwesend:} Michele Collodo, Andreas Glossner, Karl-Christoph G\"odel, Bastian Hacker, Maria Obst, Alexander Wagner, David Winnekens \\ \emph{Tutor:} Jin Xiaoyue 


\section{Protokoll des 1. Versuches} 
Die erste Version des Protokoll soll dem Tutor bis sp\"atestens Mittwoch, 11.11.2009 zugeschickt werden. Es m\"ussen bis dahin nur noch kleinere \"Anderungen am Protokoll vorgenommen werden. 

\section{Raumplanung} 
Da die R\"aume nach dem ESFZ nun wieder freigeworden sind, zieht die Gruppe f\"ur den n\"achsten Versuch wieder in den Rau 00.737 im alten PP um.

\section{Thema des 2. Projekts (4 Wochen)}
\subsection{Kerr - Effekt} Das als n\"achstes geplante Projekt \glqq Kerr-Effekt\grqq darf aufgrund eventueller Gefahren durch das zu verwendende Nitrobenzol endg\"ultig nicht in den R"aumen des Physikums durchgef\"uhrt werden. Erstaunlicher Weise haben Nachfragen zu dieser Thematik bei den verschiedenen Stellen der Praktikumsleitung zun\"achst andere Ergebnisse geliefert. Deshalb muss nun ein anderes Thema f\"ur den 2. Versuch gefunden werden.

\subsection{Messung des Kelvin Winkels}
Das neu gew\"ahlte neue Thema \glqq Kelvin Winkel\grqq  muss ebenfalls verworfen werden, da es bereits von einer anderen Praktikumsgruppe durchge\"uhrt wird.

\subsection{MHD-Generator}
Nach l\"angerer Suche wird als n\"achstes Thema der Bau eines MHD-Generators beschlossen. Der grobe Aufbau soll einen ringf\"ormigen Wasserkreislauf enthalten, an dem an einer oder mehreren Stellen ein Magnetfeld angelegt wird. Anschlie\ss end kann zwischen den beiden Seitenw\"anden des Wasserkanals an dieser Stelle eine Potentialdifferenz gemessen werden. Ver\"anderbare Parameter und M\"oglichkeiten f\"ur zus\"atzliche Messungen sind u.a. Str\"omungsgeschwindigkeit, Leitf\"ahigkeit des Wassers, Magnetfeld, Wirkungsgrad sowie der Aufbau einer Reihenschaltung

Es wird beschlossen, dass sich alle Gruppenmitglieder bis zum n\"achsten Treffen am Mittwoch \"uber das Prinzip des Effekts informieren und sich Gedanken zu einem konkreten Versuchsaufbau machen. 

\end{document}