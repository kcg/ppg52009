\documentclass[11pt]{scrartcl}
\usepackage[utf8]{inputenc} % Kodierung der Textdatei mit Sonderzeichen
\usepackage[ngerman]{babel} % Sprache fuer Inhaltsverzeichnis etc.
\usepackage{amssymb} % Mathematische Symbole
\usepackage{amsmath} % Mehr mathematische Konstrukte
\usepackage{graphicx} % Um Bilder einbinden zu koennen
\usepackage{float} % fuer \begin{figure}[H]
\usepackage{icomma} % laesst das Komma als Dezimaltrennzeichen interpretieren
\usepackage[pdftex]{hyperref} % Hyperlinks im Dokument
\hypersetup{colorlinks=true, linkcolor=black, citecolor=black, filecolor=black, urlcolor=black}


\newcommand{\unit}[1]{\ensuremath{\,\mathrm{#1}}} % Einheiten schreiben sich immer aufrecht!
\newcommand{\degr}{\ensuremath{^\circ}}
\newcommand{\cel}{\ensuremath{\degr\mathrm{C}}}
\newcommand{\dif}{\ensuremath{\mathrm{d}}}
\newcommand{\pdif}[2]{\ensuremath{\frac{\partial#1}{\partial#2}}}
\newcommand{\ee}[1]{\ensuremath{\cdot 10^{#1}}}


\setlength{\parindent}{0pt}
\setlength{\parskip}{0.5\baselineskip}


\title{Besprechungsprotokoll - PPG 5/2009}
\date{16.11.2009, 14.30 Uhr}
\author{Protokollschreiber: Maria Obst}



\begin{document}
\maketitle
\pagestyle{empty}
\section{Anwesend:} Michele Collodo, Andreas Glossner, Karl-Christoph G\"odel, Bastian Hacker, Maria Obst, Alexander Wagner, David Winnekens \\ \emph{Tutor:} Jin Xiaoyue 


\section{Protokoll des 1. Versuches} 
Das Protokoll des ersten Versuchs ist am Freitag bei Frau Anton abgegeben worden.

\section{Zweites Projekt: MHD-Generator}
\subsection{Versuchsaufbau}
Es werden Spulen anstatt Permanentmagneten f\"ur den Aufbau des Magnetfelds verwendet, da sie ein st\"arkeres und homogeneres Magnetfeld \"uber eine gr\"o\"ss{}ere Fl\"ache erzeugen. Die Pumpe ist letzte Woche angekommen und funktioniert. Die Zeichnung f\"ur die Werkstatt ist fertig und wird am selben Tag noch in Auftrag gegeben.

\subsection{Erste Messungen}
Die Leitf\"ahigkeit von Salzwasser in verschiedenen Konzentrationen bis zur S\"attigung wurde bei verschiedenen Spannungen gemessen. Diese Messung soll am Mittwoch (14:30 h) in Jin's Labor wiederholt werden, da hier bessere Messmethoden zur Verf\"ugung stehen. Daf\"ur m\"ussen noch zwei Kupferelektroden besorgt werden.

\section{Planungen f\"ur das dritte Projekt}
Erste Pl\"ane f\"ur das ditte Projekt (4 Wochen) sollen diese Woche gemacht und am n\"achsten Montag besprochen werden.

\end{document}
