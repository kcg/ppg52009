\documentclass[11pt]{scrartcl}
\usepackage[utf8]{inputenc} % Kodierung der Textdatei mit Sonderzeichen
\usepackage[ngerman]{babel} % Sprache fuer Inhaltsverzeichnis etc.
\usepackage{amssymb} % Mathematische Symbole
\usepackage{amsmath} % Mehr mathematische Konstrukte
\usepackage{graphicx} % Um Bilder einbinden zu koennen
\usepackage{float} % fuer \begin{figure}[H]
\usepackage{icomma} % laesst das Komma als Dezimaltrennzeichen interpretieren
\usepackage[pdftex]{hyperref} % Hyperlinks im Dokument
\hypersetup{colorlinks=true, linkcolor=black, citecolor=black, filecolor=black, urlcolor=black}


\newcommand{\unit}[1]{\ensuremath{\,\mathrm{#1}}} % Einheiten schreiben sich immer aufrecht!
\newcommand{\degr}{\ensuremath{^\circ}}
\newcommand{\cel}{\ensuremath{\degr\mathrm{C}}}
\newcommand{\dif}{\ensuremath{\mathrm{d}}}
\newcommand{\pdif}[2]{\ensuremath{\frac{\partial#1}{\partial#2}}}
\newcommand{\ee}[1]{\ensuremath{\cdot 10^{#1}}}


\setlength{\parindent}{0pt}
\setlength{\parskip}{0.5\baselineskip}


\title{Besprechungsprotokoll - PPG 5/2009}
\date{23.11.2009, 14.30 Uhr}
\author{Protokollschreiber: Karl Christoph Gödel}



\begin{document}
\maketitle
\pagestyle{empty}
\section{Anwesend:} Michele Collodo, Andreas Glossner, Karl-Christoph G\"odel, Bastian Hacker, Maria Obst, Alexander Wagner, David Winnekens \\ \emph{Tutor:} Jin Xiaoyue 


\section{Aktuelles Projekt: MHD-Generator}
\subsection{Versuchsaufbau und erste Messung}
Der Versuchsaufbau steht und ist funktionsfähig. Erste Messungen der erzeugten Spannung stimmen sehr gut mit den vorher berechneten theoretischen Abschätzungen überein. Der MHD-Generator liefert eine Spannung ohne Fluß und B-Feld, deren Ursache noch zu ermitteln ist.

\subsection{Nächste Messungen}
Die Leitf\"ahigkeit von Salzwasser in diesem Aufbau soll mit Hilfe der zweiten Plexiglasröhre bestimmt werden.

Der Grund für die Spannung ohne Fluß und B-Feld soll herausgefunden werden.

Die Abhängigkeit von B-Feld und generierter Spannung soll ermittelt werden. Bei guten Ergebnissen diesbezüglich, ist über die Anschaffung einer stärkeren Pumpe nachzudenken.

\section{Planungen f\"ur das dritte Projekt}
Es werden etwa 60 LEDs benötigt (viele verschiedene Spektralbereiche). Die Anschaffungskosten werden im Bereich von 60-70 Euro liegen.

Es soll parallel zum MHD-Projekt, in dieser Woche, mit den Vorbereitungen für das LED-Spektrum-Projekt begonnen werden.

\section{Foto für die PPG5-Homepage}
Das Foto wurde vor den Univeritätsgebäuden der Physik gemacht und auf die Homepage geladen.

\end{document}
