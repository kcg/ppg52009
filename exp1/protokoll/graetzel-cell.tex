\documentclass[11pt]{scrartcl}
\usepackage[utf8]{inputenc} % Kodierung der Textdatei mit Sonderzeichen
%\usepackage[ngerman]{babel} % Sprache für Inhaltsverzeichnis etc.
\usepackage{amssymb} % Mathematische Symbole
\usepackage{amsmath} % Mehr mathematische Konstrukte
\usepackage{graphicx} % Um Bilder einbinden zu können
\usepackage{float} % für \begin{figure}[H]



\newcommand{\unit}[1]{\ensuremath{\,\mathrm{#1}}} % Einheiten schreiben sich immer aufrecht!
\newcommand{\degr}{\ensuremath{^\circ}}
\newcommand{\cel}{\ensuremath{\degr\mathrm{C}}}
\newcommand{\dif}{\ensuremath{\mathrm{d}}}
\newcommand{\pdif}[2]{\ensuremath{\frac{\partial#1}{\partial#2}}}
\newcommand{\ee}[1]{\ensuremath{\cdot 10^{#1}}}



\title{Graetzel cell - Gruppe 5 WS 09/10, Projektpraktikum der Uni Erlangen}
\date{19.10.2009 -- 30.11.2009}
\author{Michele Collodo, Andreas Glossner, Karl-Christoph Gödel, Bastian Hacker, Maria Obst, Alexander Wagner, David Winnekens}



\begin{document}
\sloppy % lässt Latex nicht über den Rand rausschreiben
\thispagestyle{empty}
\large{Projektpraktikum WS 09/10}
\hfill
\raisebox{-1.4cm}{\includegraphics[width=5cm]{images/fau.pdf}}
\\[8\baselineskip]
\begin{center}
\Huge{\textbf{Graetzel cell}}\\[0.5\baselineskip]
\Large{19.10.2009 -- 30.11.2009}
\\[6\baselineskip]
\Huge{\textbf{PPG 5}}\\[0.5\baselineskip]
\large{\textbf{Michele Collodo, Andreas Glossner,\\
Karl-Christoph Gödel, Bastian Hacker,\\
Maria Obst, Alexander Wagner, David Winnekens}\\
Tutor: Xiaoyue Jin}
\vfill



\small{\texttt{http://pp.physik.uni-erlangen.de/groups/ws0910/ppg5/ppg5\_start.html}}
\end{center}
\newpage



\tableofcontents
\vfill



\begin{abstract}
Description of our work \ldots
\end{abstract}
\newpage



\section{introduction}
Content and stuff \ldots


\end{document}

